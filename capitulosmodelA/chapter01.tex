\artigofalse
\chapter{A FILOSOFIA DA MATEMÁTICA DE FREGE}

O objetivo deste capítulo é descrever e discutir alguns pontos da filosofia da matemática de Frege, bem como apresentar alguns resultados matemáticos desenvolvidos na terceira parte de \textit{Begriffsschrift} (1879\nocite{Frege1998}) e nas §§46-83 de \textit{Die Grundlagen der Arithmetik}.

\section{LOGICISMO}

Entre inúmeras tendências de pensamento na matemática no século XIX, uma das principais foi o movimento fundacionalista\footnote{O movimento fundacionalista, como entendemos, foi a tentativa, por parte dos matemáticos, de fornecer os fundamentos mais seguros e racionais para sua ciência.}. Muitos foram os matemáticos que exigiam um maior rigor nas definições de conceitos matemáticos e nas provas de teoremas. O movimento marcou também o rompimento entre a geometria e a aritmética. As definições de conceitos aritméticos tinham de ser explicados por meio de outros conceitos aritméticos mais básicos. Segundo alguns comentadores, por exemplo, Demopoulos (1994 \nocite{Demopoulos1995a}\nocite{Demopoulos1995}), a rigorização da matemática e o rompimento entre geometria e aritmética\footnote{Por aritmética aqui, entendemos a aritmética dos números naturais e análise real. Quando nos referirmos apenas à aritmética dos números naturais, designaremos da seguinte forma: aritmética dos números naturais.} assinalavam uma transformação nas idéias dos matemáticos, a saber, que a aritmética formava uma ciência independente. Em última análise, se a aritmética dependesse da geometria para explicar seus conceitos, então a aritmética dependeria dos conceitos de tempo e espaço\footnote{Uma outra razão para o rigor era garantir a consistência e coerência da análise.} 

\begin{quote}
Neste aspecto o combate à incursão da intuição Kantiana, as motivações intelectuais de Frege refletem as dos analistas do século XIX que buscavam livrar o cálculo e a teoria dos reais de qualquer dependência da geometria e cinemática. Assim, já em 1817 Bolzano escreveu: ‘os conceitos de tempo e movimento são tão estranhos à matemática geral quanto o conceito de espaço’.\cite[p. 76]{Demopoulos1995a}
\end{quote}
