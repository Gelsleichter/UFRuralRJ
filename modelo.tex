%\listfiles

%% Tipo de documento e a classe a ser usada para sua formatação.
\documentclass[tese]{UFRuralRJ}

%% Um tipo específico de monografia pode ser informado como parâmetro opcional:
%\documentclass[tese]{UFRuralRJ}

%% A opção 'openright' força inícios de capítulos em páginas ímpares
%\documentclass[openright]{UFRuralRJ}

%% Use a opção 'twoside' para gerar uma versão frente-e-verso
%\documentclass[twoside]{UFRuralRJ}

%%==============================================================================
%% Pacotes - língua, codificação e fonte
%%==============================================================================

\usepackage[brazilian]{babel}
\usepackage[T1]{fontenc}              %% Conjunto de caracteres correto
\usepackage{times, color, xcolor}     %% Usar fonte Adobe Times e cores
\usepackage[utf8]{inputenc}           %% Acentuação

%%==============================================================================
%% Pacotes - formatação de equações e números
%%==============================================================================

\usepackage{amsmath,latexsym,amssymb}
\usepackage{siunitx}                  %% Sistema Internacional de Unidades

%%==============================================================================
%% Pacotes - formatação de figuras
%%==============================================================================

\usepackage{graphicx}                 %% Importar figuras
\graphicspath{{capitulos/figuras/}}   %% Diretório onde estão as figuras
\usepackage{float}
\usepackage{wrapfig}

%%==============================================================================
%% Pacotes - formatação de hyperlinks
%%==============================================================================
%% Opção 'hidelinks' disponível no pacote 'hyperref' a partir da versão 
%% 2011-02-05  6.82a. 'hidelinks' retira os retângulos do entorno das palavras
%% com links.

\usepackage[%hidelinks%, 
            bookmarksopen=true,linktoc=none,colorlinks=true,
            linkcolor=blue,citecolor=blue,filecolor=magenta,urlcolor=blue,
            pdftitle={Este título é muito legal},
            pdfauthor={Nome do Melhor Estudante da Rural},
            pdfsubject={Tese de Doutorado},
            pdfkeywords={LaTeX, UFRuralRJ, Documentos acadêmicos}
            ]{hyperref}

%%TODO: Margens conforme MDT UFSM 7ª edição. Corrigir no arquivo UFRuralRJ.cls 
%%      para funcionar a opção twoside *PENDENTE*
\usepackage[inner = 30mm, outer = 20mm, top = 30mm, bottom = 20mm]{geometry}
%% Se o pacote 'hyperref' acima foi carregado, a linha abaixo corrige um bug na 
%% hora de montar o sumário da lista de figuras e tabelas. Comente a linha se o
%% pacote 'hyperref' não foi carregado.
\input{macros/bugcaption}

%%==============================================================================
%% Pacotes - formatação de verbatim
%%==============================================================================
%% O ambiente verbatim é o ambiente onde são inseridos exemplos de código fonte.
%% Está opção adiciona cor de fundo ao ambiente verbatim.
%% Comente para desabilitar.

\usepackage{color}
\let\oldv\verbatim
\let\oldendv\endverbatim
\def\verbatim{\par\setbox0\vbox\bgroup\oldv}
\def\endverbatim{\oldendv\egroup\fboxsep0pt 
                 \noindent\colorbox[gray]{0.8}{\usebox0}\par}

%%==============================================================================
%% Pacotes - outros
%%==============================================================================

\usepackage{blindtext}                %% Amostra de texto (\blindtext[1])
\usepackage{fix-cm}                   %% Tamanho das fontes da capa

%%==============================================================================
%% Identificação do trabalho
%%==============================================================================

\title{Este título é muito legal}                              %% Título em português
\titleENG{This is a cool title}                                %% Título em inglês
\author{Rural}{Nome do Melhor Estudante da}                    %% Autor: sobrenome, nome
%\autoratrue                                                   %% Descomentar se for uma AUTORA
\institute{Instituto da Agronomia}                             %% Instituto
\course{Curso de Pós-Graduação em Agronomia - Ciência do Solo} %% Curso de pós-graduação
\departamento{Departamento de Solos}                           %% Departamento
\altcourse{Curso de Pós-Graduação em Agronomia}                %% Curso de pós-graduação sem a área de concentração
\concentra{Ciência do Solo}                                    %% Área de concentração
\degree{Doutor em Ciências}                                    %% Grau obtido (geral)
\altdegree{Doutorado em Agronomia, Ciência do Solo}            %% Grau obtido (específico)
\altdegreeENG{Doctor of Science in Agronomy, Soil Science}     %% Grau obtido (específico) em inglês
\trabalhoNumero{}                                              %% Número do TG - deixar em branco para mestrado

%%==============================================================================
%% Identificação dos orientadores
%%==============================================================================

\advisor[Professor]{Dr.}{Rural}{Nome do Melhor Orientador da}{UFRRJ}  %% Orientador
%\orientadoratrue                                                     %% Descomentar se for uma ORIENTADORA
\coadvisor[Professor]{Dr.}{Rural}{Nome do Melhor Co-orientador da}    %% Co-orientador
\coadvisor[Professora]{Dra.}{Rural}{Nome da Melhor Co-orientadora da} %% Co-orientadora
%% Descomentar o comando abaixo se for UMA CO-ORIENTADORA
%% No caso de DOIS OU MAIS co-orientadores, descomentar apenas quando TODOS forem CO-ORIENTADORAS
%\coorientadoratrue
\coorientadorestrue                                                   %% Descomentar se forem DOIS OU MAIS CO-ORIENTADOR(A)
%%==============================================================================
%% Informações sobre a defesa
%%==============================================================================
\committee[Dra.]{Rural}{Nome da Melhor Examinadora da}{UFRRJ}          %% Examinador 1
\committee[Dr.]{Rural}{Nome do Melhor Examinador da}{UFRRJ}            %% Examinador 2
\committee[Dra.]{Rural}{Nome da Melhor Examinadora de Fora da}{MEFR}   %% Examinador 3
\committee[Dr.]{Rural}{Nome do Melhor Examinador de Fora da}{MEFR}     %% Examinador 4
\date{30}{Fevereiro}{2016}                                             %% Data da defesa
%%==============================================================================
%% Outros itens
%%==============================================================================
\keyword{LaTeX}                 %% Palavra chave 1 (português)
\keyword{UFRuralRJ}             %% Palavra chave 2 (português)
\keyword{Documentos acadêmicos} %% Palavra chave 3 (português)
%%==============================================================================
%% Início do documento
%%==============================================================================
\begin{document}
%%==============================================================================
%% Capa e folha de rosto
%%==============================================================================
\maketitle
%%==============================================================================
%% Ficha catalográfica
%%==============================================================================
%% Como a CIP vai ser impressa atrás da página de rosto, as margens 'inner' e 
%% 'outer' devem ser invertidas.
%\newgeometry{inner=20mm,outer=30mm,top=30mm,bottom=20mm}
%\makeCIP{alessandrosamuel@yahoo.com.br} %% email do autor		
%\restoregeometry
%Se for usar a catalogação gerada pelo gerador do site da biblioteca comentar as linhas
%acima e utilizar o comando abaixo
%\includeCIP{CIP.pdf}

%%=============================================================================
% Folha de aprovação
%%=============================================================================

\makeapprove

%%=============================================================================
%% Dedicatória (opcional)
%%=============================================================================

\clearpage\mbox{}\vfill\hspace{80mm}
\begin{minipage}{76mm}
  \begin{flushright}
    {\em
    Àqueles que financiaram meus estudos...
    \par
    ...DEDICO!
    }
  \end{flushright}
\end{minipage}

%%=============================================================================
%% Agradecimentos (opcional)
%%=============================================================================

\chapter*{Agradecimentos}

Ao Curso de Pós-Graduação em Agronomia-Ciência do Solo (CPGA-CS) e ao 
Departamento de Solos da Universidade Federal Rural do Rio de Janeiro 
(UFRuralRJ).

Às agências de fomento.

Aos orientadores.

À todos que, direta ou indiretamente, contribuíram para a construção deste 
trabalho.

%%=============================================================================
%% Biografia (opcional)
%%=============================================================================

\chapter*{Biografia}
O autor nasceu, cresceu e escreveu uma tese.

%%=============================================================================
%% Epígrafe (opcional)
%%=============================================================================

\clearpage\mbox{}\vfill\hspace{80mm}\begin{minipage}{76mm}\begin{flushright}{\em
``Fazer é a melhor forma de dizer.''
\par
Autor desconhecido
}\end{flushright}\end{minipage}

%%==============================================================================
%% Resumo geral (português)
%%==============================================================================

\generalabstracttrue
\begin{abstract}
Este é o resumo em português de minha tese. Claramente, este é o melhor resumo
que já foi escrito em um documento acadêmico produzido na UFRuralRJ.
\end{abstract}

%%==============================================================================
%% General abstract (inglês)
%%==============================================================================
\generalabstracttrue
\begin{englishabstract}
{This is a cool title}                                   %% Título do trabalho
{Curso de Pós-Graduação em Agronomia - Ciência do Solo}  %% Curso de pós-graduação
{LaTeX, UFRuralRJ, Academical documents}                 %% Palavras-chave
{November}                                               %% Mês
{th}                                                     %% Sigla do dia
This is the English abstract of my thesis. Obviously this is the best abstract 
that has ever been written in an academic document produced at UFRuralRJ.
\end{englishabstract}
%%=============================================================================
%% Listas (comentar se não houver) e sumário
%%=============================================================================
\listoffigures                         %% Lista de figuras
%\listoftables                         %% Lista de tabelas
\listofappendix                        %% Lista de apêndices
%\listofannex                          %% Lista de anexos
%\begin{listofabbrv}{UbiComp}          %% Lista de abreviaturas e siglas
%   \item [DSM] Digital Soil Mapping
%   \item [UbiComp] Computação Ubíqua
%\end{listofabbrv}
%\begin{listofsymbols}{teste}          %% Lista de simbolos (opcional)
%  \item [$\varnothing$] vazio         %% Simbolos devem aparecer conforme a ordem em que aparecem no texto
%  \item [$\Gamma$]  Gama              %% O parâmetro deve ser o símbolo mais longo
%  \item [$\forall$] Para todo
%\end{listofsymbols}
\tableofcontents                       %% Sumário
%%=============================================================================
%% Início da tese
%%=============================================================================
\setlength{\baselineskip}{1.5\baselineskip}
\setcounter{page}{1}
\artigofalse
\chapter{Introdução}
\label{chap:introduction}

\blindtext[2]

\begin{figure}[!ht]
\centering
\includegraphics[width=16cm]{figura01}
\caption{Jardim Botânico da UFRuralRJ. Fonte: \url{http://commons.wikimedia.org/wiki/File:Jardim_Bot\%C3\%A2nico_UFRRJ.jpg}}
\label{fig:jardim}
\end{figure}

\blindtext[1]

\section{SEÇÃO}

\blindtext[2]

\subsection{Subseção}

\blindtext[2] %% Incluir capítulo 00
\artigotrue
\chapter{Título do Primeiro artigo}
\label{chap:chapter01}

\begin{chapterabstract}{portuguese}{Palavra-chave 1, Palavra-chave 2, Palavra-chave 3}
\noindent{Este é o resumo do primeiro artigo da tese.}
\end{chapterabstract}

\begin{chapterabstract}{english}{Key-word 1, Key-word 2, Key-word 3}
\noindent{This is the abstract of the first article of the thesis.}
\end{chapterabstract}

\formatchapter

\section{INTRODUÇÃO}

\blindtext[2]

\section{MATERIAL E MÉTODOS}

Este é um texto bem formatado, escrito em Seropédica, RJ. \blindtext[1]

Este é o código fote de uma função construída no ambiente R:

\begin{verbatim}
> soma <- function (a, b) {a + b}
> soma(2, 2)
[1] 4
\end{verbatim}

Está é uma matriz bem formatada:

\begin{equation}
  A_{m,n} =
 \begin{pmatrix}
  a_{1,1} & a_{1,2} & \cdots & a_{1,n} \\
  a_{2,1} & a_{2,2} & \cdots & a_{2,n} \\
  \vdots  & \vdots  & \ddots & \vdots  \\
  a_{m,1} & a_{m,2} & \cdots & a_{m,n}
 \end{pmatrix}
\end{equation}

\begin{subequations}\label{eq:maxwell}
E estas são as equações de Maxwell:
\begin{align}
        B'&=-\nabla \times E,\\
        E'&=\nabla \times B - 4\pi j,
\end{align}
\end{subequations}

\section{RESUTADOS}

Aqui está mais um texto bem formatado. \blindtext[1]

Que tal fazer um link para a figura \autoref{fig:ocio}? E também citar o \citet{Feyerabend1977} com um link para a localização da referência bibliográfica?

\begin{figure}[!ht]
\centering
\includegraphics[width=16cm]{figura02}
\caption{\label{fig:ocio}O ócio criativo. Fonte: \url{http://r1.ufrrj.br/graduacao/img/acesso-2012/o-ocio-criativo.jpg}}
\end{figure}

\section{DISCUSSÃO}

Aqui está o último texto muito bem formatado. \blindtext[2]

Que tal fazer um link para a equação \autoref{eq:maxwell}?

\section{CONCLUSÕES}

\begin{itemize}
  \item Está é uma conclusão importante.
  \item Está é outra conclusão importante.
  \item Está é uma conclusão menos importante.
\end{itemize}
 %% Incluir capítulo 01
\artigotrue
\chapter{Título do segundo artigo}
\label{chap:chapter01}

\begin{chapterabstractPOR}{Palavra-chave 1, Palavra-chave 2, Palavra-chave 3}
Este é o resumo do segundo artigo da tese. Reconheço que este artigo não é muito
diferente do anterior... Mas quem se importa?
\end{chapterabstractPOR}

\begin{chapterabstractENG}{Key-word 1. Key-word 2. Key-word 3}
This is the abstract of the second article of the thesis. I recognize that it is
not very different from the previous... But who cares?
\end{chapterabstractENG}

\section{INTRODUÇÃO}

\blindtext[2]

\section{MATERIAL E MÉTODOS}

Este também é um texto bem formatado, escrito em Seropédica, RJ. \blindtext[1]

Este é o código fonte de uma função muito complexa construída no ambiente R:

\begin{verbatim}
> soma <- function (a, b) {a + b}
> soma(2, 2)
[1] 4
\end{verbatim}

Está é uma matriz bem formatada, diferente daquelas produzidas pelos editores de
texto tradicionais:

\begin{equation}
  A_{m,n} =
 \begin{pmatrix}
  a_{1,1} & a_{1,2} & \cdots & a_{1,n} \\
  a_{2,1} & a_{2,2} & \cdots & a_{2,n} \\
  \vdots  & \vdots  & \ddots & \vdots  \\
  a_{m,1} & a_{m,2} & \cdots & a_{m,n}
 \end{pmatrix}
\end{equation}

\begin{subequations}\label{eq:maxwell2}
E estas são as equações de Maxwell (sim, de Maxwell!):
\begin{align}
        B'&=-\nabla \times E,\\
        E'&=\nabla \times B - 4\pi j,
\end{align}
\end{subequations}

\section{RESUTADOS}

Para quem ainda não está cansado, aqui está mais um texto bem formatado. 
\blindtext[1]

Que tal fazer um link para a \autoref{fig:ocio2}? E também citar o 
\citet{Feyerabend1977} com um link para a localização da referência 
bibliográfica? Sim, isso já foi feito antes!

\begin{figure}[!ht]
\centering
\includegraphics[width=16cm]{figura02}
\caption{\label{fig:ocio2}O ócio criativo. Fonte: 
\url{http://r1.ufrrj.br/graduacao/img/acesso-2012/o-ocio-criativo.jpg}}
\end{figure}

\section{DISCUSSÃO}

Aqui está o último texto muito bem formatado. \blindtext[2]

Que tal fazer um link para a \autoref{eq:maxwell2}?

\section{CONCLUSÕES}

\begin{itemize}
  \item Está é uma conclusão importante.
  \item Está é outra conclusão importante.
  \item Está é uma conclusão menos importante.
\end{itemize}
 %% Incluir capítulo 02
\artigofalse
\chapter{CONCLUSÃO GERAL}
\shorttitle{Conclusão geral}
\label{chap:conclusion}

\blindtext[2]

\blindtext[2]

\blindtext[2]
 %% Incluir capítulo 03
\include{capitulos/chapter04} %% Incluir capítulo 04
\include{capitulos/chapter05} %% Incluir capítulo 05
\setlength{\baselineskip}{\baselineskip}
%%=============================================================================
%% Referências
%%=============================================================================
\bibliographystyle{abnt}
\bibliography{referencias/biblio}
%%=============================================================================
%% Apêndices e anexos
%% Se precisar usar alguma seção ou subseção dentro dos apêndices ou
%% anexos, utilizar o comando \tocless para não adicionar no Sumário
%% Exemplos: 
%% \tocless\section{Histórico}
%%=============================================================================
\appendix                      %% Apêndices
\artigofalse
\chapter{Primeiro apêndice}
\label{anex:apendiceA}

\tocless\section{Introdução}

\blindtext[2]

\tocless\section{Subseção}

\blindtext[2]
  %% Incluir apêndice A
%\include{capitulos/apendiceb} %% Incluir apêndice B
%\annex                        %% Anexos
%\include{capitulos/anexoa}    %% Incluir anexo A
\end{document}