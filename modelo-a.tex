%%%%%%%%%%%%%%%%%%%%%%%%%%%%%%%%%%%%%%%%%%%%%%%%%%%%%%%%%%%%%%%%%%%%%%%%%%%%%%%%%%%%%
%Modelo para dissertação - Humanas
%%%%%%%%%%%%%%%%%%%%%%%%%%%%%%%%%%%%%%%%%%%%%%%%%%%%%%%%%%%%%%%%%%%%%%%%%%%%%%%%%%
%\listfiles

%% Tipo de documento e a classe a ser usada para sua formatação.
\documentclass[diss]{UFRuralRJ}

%% Um tipo específico de monografia pode ser informado como parâmetro opcional:
%\documentclass[tese]{UFRuralRJ}
%% A opção 'openright' força inícios de capítulos em páginas ímpares
%\documentclass[openright]{UFRuralRJ}

%% Use a opção 'twoside' para gerar uma versão frente-e-verso
%\documentclass[twoside]{UFRuralRJ}

%%%%%%%%%%%%%%%%%%%%%%%%%%%%%%%%%%%%%%%%%%%%%%%%%%%%%%%%%%%%%%%%%%%%%%%%%%%%%%
%Pacote para gerar bibliografia segundo NBR6023
%%%%%%%%%%%%%%%%%%%%%%%%%%%%%%%%%%%%%%%%%%%%%%%%%%%%%%%%%%%%%%%%%%%%%%%%%%%%%%%
%\usepackage[alf]{abntex2cite}
%%==============================================================================
%% Pacotes - língua, codificação e fonte
%%==============================================================================

\usepackage[brazilian]{babel} %% use 'english' ao invés de 'brazilian' para documento escrito em inglês
\usepackage[T1]{fontenc} %% Conjunto de caracteres correto
%\usepackage{times} %% Usar fonte Adobe Times Roman, equivalente à Times New Roman
\usepackage[utf8]{inputenc} %% Para acentuação correta

%%==============================================================================
%% Pacotes - formatação de equações e números
%%==============================================================================

\usepackage{amsmath,latexsym,amssymb}
\usepackage{siunitx}                  %% Sistema Internacional de Unidades

%%==============================================================================
%% Pacotes - formatação de figuras
%%==============================================================================

%% Importar figuras corretamente
\usepackage{graphicx}

%% Diretório onde estão as figuras dos capítulos
\graphicspath{{capitulos/figuras/}}

\usepackage{float}
\usepackage{wrapfig}

%%==============================================================================
%% Pacotes - formatação de hyperlinks
%%==============================================================================
%% Opção 'hidelinks' disponível no pacote 'hyperref' a partir da versão 
%% 2011-02-05  6.82a. 'hidelinks' retira os retângulos do entorno das palavras
%% com links.



\usepackage[%hidelinks%, 
           bookmarksopen=true,linktoc=none,colorlinks=true,
           linkcolor=blue,citecolor=blue,filecolor=magenta,urlcolor=blue,
            pdftitle={Este título é muito legal para uma tese escrita em português},
            pdfauthor={Nome do Melhor Estudante da Rural},
            pdfsubject={Tese de Doutorado},
            pdfkeywords={LaTeX, UFRuralRJ, Documentos acadêmicos}
            ]{hyperref}

%%TODO: Margens conforme MDT UFSM 7ª edição. Corrigir no arquivo UFRuralRJ.cls 
%%      para funcionar a opção twoside *PENDENTE*
\usepackage[inner = 30mm, outer = 20mm, top = 30mm, bottom = 20mm]{geometry}
%% Se o pacote 'hyperref' acima foi carregado, a linha abaixo corrige um bug na 
%% hora de montar o sumário da lista de figuras e tabelas. Comente a linha se o
%% pacote 'hyperref' não foi carregado.

%%=============================================================================
%% Trampa para corrigir o bug do hyperref que redefine o caption das figuras e das
%% tabelas, n�o colocando o nome ``Figura'' antes do n�mero do mesmo na lista
%%=============================================================================

\makeatletter

\long\def\@caption#1[#2]#3{%
  \expandafter\ifx\csname if@capstart\expandafter\endcsname
                  \csname iftrue\endcsname
    \global\let\@currentHref\hc@currentHref
  \else
    \hyper@makecurrent{\@captype}%
  \fi
  \@ifundefined{NR@gettitle}{%
    \def\@currentlabelname{#2}%
  }{%
    \NR@gettitle{#2}%
  }%
  \par\addcontentsline{\csname ext@#1\endcsname}{#1}{%
    \protect\numberline{\csname fnum@#1\endcsname ~-- }{\ignorespaces #2}%
  }%
  \begingroup
    \@parboxrestore
    \if@minipage
      \@setminipage
    \fi
    \normalsize
    \expandafter\ifx\csname if@capstart\expandafter\endcsname
                    \csname iftrue\endcsname
      \global\@capstartfalse
      \@makecaption{\csname fnum@#1\endcsname}{\ignorespaces#3}%
    \else
      \@makecaption{\csname fnum@#1\endcsname}{%
        \ignorespaces
        \ifHy@nesting
          \expandafter\hyper@@anchor\expandafter{\@currentHref}{#3}%
        \else
          \Hy@raisedlink{%
            \expandafter\hyper@@anchor\expandafter{%
              \@currentHref
            }{\relax}%
          }%
          #3%
        \fi
      }%
    \fi
    \par
  \endgroup
}

\makeatother

%%==============================================================================
%% Pacotes - formatação de verbatim
%%==============================================================================
%% O ambiente verbatim é o ambiente onde são inseridos exemplos de código fonte.
%% Está opção adiciona cor de fundo ao ambiente verbatim.
%% Comente para desabilitar.

\let\oldv\verbatim
\let\oldendv\endverbatim
\def\verbatim{\par\setbox0\vbox\bgroup\oldv}
\def\endverbatim{\oldendv\egroup\fboxsep0pt 
                 \noindent\colorbox[gray]{0.8}{\usebox0}\par}

%%==============================================================================
%% Pacotes - outros
%%==============================================================================

\usepackage{blindtext}                %% Amostra de texto (\blindtext[1])
%\usepackage{fix-cm} %% Tamanho das fontes da capa (já é carregado pelo pacote)

%%==============================================================================
%% Identificação do trabalho
%%==============================================================================

\title{Princípio de Hume: Possibilidade de uma Filosofia (Neo) Fregeana da Aritmética?} %% Título em português
\titleENG{Hume's Principle: Possibility of a (Neo) Fregean Philosophy of Arithmetic?} %% Título em inglês
\author{Duarte}{Alessandro Bandeira} %% Autor: sobrenome, nome
%\autoratrue %% No caso de uma AUTORA
\instituto{Instituto de Ciências Humanas e Sociais} %% Instituto
\curso{Curso de Pós-Graduação em Filosofia} %% Curso de pós-graduação (com a área de concentração?)
                                        %% Ex.: Curso de Pós-Graduação em Agronomia - Ciência do Solo
%\altCurso{Curso de Pós-Graduação em Filosofia} %% Curso de pós-graduação sem a área de concentração
\area{Filosofia} %% Área de concentração.
\grau{Mestre em Filosofia} %% Grau obtido (geral)
\altGrau{Mestrado em Filosofia} %% Grau obtido (específico). Ex.: Doutorado em Agronomia, Ciência do Solo
\altGrauENG{Master of Philosophy} %% Grau obtido (específico) em inglês. Ex.: Doctor of Science in Agronomy, Soil Science
\trabalhoNumero{} %% Número do trabalho de graduação - deixar em branco para mestrado
\local{Serop\'edica}{RJ}{Brasil} %% Cidade, estado, país

%%==============================================================================
%% Identificação dos orientadores
%%==============================================================================

\advisor[Professor]{Dr.}{Dedekind}{Richard}{UFRRJ} %% Orientador
%\orientadoratrue %% Descomentar se for uma ORIENTADORA
%\coadvisor[Pesquisador]{Dr.}{Rural}{Melhor Co-orientador da} %% Co-orientador
%\coadvisor[Professora]{Dra.}{Rural}{Melhor Co-orientadora da} %% Co-orientadora

%% COMANDOS OBSOLETOS
%% Descomentar o comando abaixo se for UMA CO-ORIENTADORA
%% No caso de DOIS OU MAIS co-orientadores, descomentar apenas quando todos forem CO-ORIENTADORAS
%\coorientadoratrue 
%\coorientadorestrue %% dois ou mais co-orientadores(as)

%%==============================================================================
%% Informações sobre a defesa
%%==============================================================================

% Ex.: \committee[Título]{Sobrenome}{Nome}{Instituição}
\committee[Dr.]{Dedekindbr}{Richard}{UFRRJ} %% Presidente
\committee[Dr.]{Frege}{Gottlob}{UFRRJ} %% Examinador
\committee[Dra.]{Maddy}{Peneloppy}{UFRGT} %% Examinador
%\committee[Dra.]{Banca}{Outra Melhor Integrante da}{MEFR} %% Examinador
%\committee[Dr.]{Banca}{Outro Melhor Integrante da}{MEFR} %% Examinador
\date{30}{Fevereiro}{2016} %% Data da defesa

%%==============================================================================
%% Outros itens
%%==============================================================================

\keyword{Frege}                 %% Palavra chave 1 (português)
\keyword{Princípios de Abstração}             %% Palavra chave 2 (português)
\keyword{Princípio de Hume} %% Palavra chave 3 (português)
%\keyword{teste}
%%==============================================================================
%% Início do documento
%%==============================================================================
\begin{document}
%%==============================================================================
%% Capa e folha de rosto
%%==============================================================================
\maketitle

%%==============================================================================
%% Ficha catalográfica
%%==============================================================================
%% Como a CIP vai ser impressa atrás da página de rosto, as margens 'inner' e 
%% 'outer' devem ser invertidas.
%\newgeometry{inner=20mm,outer=30mm,top=30mm,bottom=20mm}
%\makeCIP{alessandrosamuel@yahoo.com.br} %% email do autor		
%\restoregeometry
%Se for usar a catalogação gerada pelo gerador do site da biblioteca comentar as
%linhas acima e utilizar o comando abaixo
%\includeCIP{CIP.pdf}

%%=============================================================================
% Folha de aprovação
%%=============================================================================
\makeapprove

%%=============================================================================
%% Dedicatória (opcional)
%%=============================================================================

%\clearpage\mbox{}\vfill\hspace{80mm}
%\begin{minipage}{76mm}
%  \begin{flushright}
%    {\em
%    Àqueles que financiaram meus estudos...
%    \par
%    ...DEDICO!
%    }
%  \end{flushright}
%\end{minipage}

%%=============================================================================
%% Agradecimentos (opcional)
%%=============================================================================

\chapter*{Agradecimentos}

\noindent Agradeço à minha família que me apoiou em todos os momentos difíceis;\\ 
à Eleonora, pela paciência e carinho; \\
a Stefano Stival, Michael Pontes e Luciano da Silva pelas nossas conversas informais que edificaram muitas idéias; \\
a Flávio Esteves, por sua grande amizade; \\
ao meu grande amigo José Rubens que mesmo distante sempre teve uma palavra amiga; \\
aos Profs. Drs. Luiz Carlos P. D. Pereira e Danilo Marcondes de Souza Filho que se dispuseram a participar da Banca Examinadora; \\
ao Prof. Dr. Marco Ruffino, pela sua amizade; \\
aos Profs. Drs. Richard Heck, Christian Thiel e Gottfried Gabriel pela colaboração e atenção sobre a tradução da carta de Frege a Russell (28/7/1902); \\
finalmente, ao meu orientador Prof. Dr. Oswaldo Chateaubriand pela sua pela orientação e paciência.

%%=============================================================================
%% Biografia (opcional)
%%=============================================================================

%\chapter*{Biografia}
%O autor nasceu, cresceu e escreveu uma tese.

%%=============================================================================
%% Epígrafe (opcional)
%%=============================================================================

\clearpage\mbox{}\vfill\hspace{80mm}\begin{minipage}{76mm}\begin{flushright}{\em
``Fazer é a melhor forma de dizer.''
\par
Autor desconhecido
}\end{flushright}\end{minipage}

%%==============================================================================
%% Resumo geral (português)
%%==============================================================================

%\generalabstracttrue
\begin{abstract}
A dissertação apresenta e discute as idéias desenvolvidas por Crispin Wright no livro \textit{Frege’s Conception of Numbers as Objects} (1983), em particular, a tese segundo a qual a aritmética é analítica. Wright deposita toda sua força argumentativa (em relação à analiticidade da aritmética) na derivação dos axiomas da aritmética de segunda ordem de Dedekind-Peano a partir do Princípio de Hume. Assim, é nosso principal objetivo apresentar e discutir em que medida o Princípio de Hume é capaz de fornecer, segundo Wright, um relato da analiticidade da aritmética, assim como, as objeções a esse relato.
\end{abstract}

%%==============================================================================
%% General abstract (inglês)
%%==============================================================================

%\generalabstracttrue
\begin{englishabstract}
{Hume's Principle: Possibility of a (Neo) Fregean Philosophy of Arithmetic?} %% Título do trabalho
{Curso de Pós-Graduação em Filosofia} %% Curso de pós-graduação
{Frege, Abstraction Principles, Hume's Principle} %% Palavras-chave
{November} %% Mês
{th} %% Sigla do dia
The dissertation presents and discusses the ideas developed by Crispin Wright in his book \textit{Frege's Conception of Numbers as Objects} (1983), in particular his thesis that arithmetic is analytic. Wright concentrates all his argumentative efforts (in relation to the analyticity of arithmetic) on the derivation of the axioms of Dedekind-Peano's second order arithmetic from Hume's Principle. Thus, it is our main goal to present and discuss how Hume's Principle provides, according to Wright, an explanation of the analytic character of arithmetic as well as some objections to this account.
\end{englishabstract}

%%=============================================================================
%% Listas (comentar se não houver) e sumário
%%=============================================================================

%\listoffigures %% Lista de figuras
%\listoftables %% Lista de tabelas
%\listofappendix %% Lista de apêndices
%\listofannex %% Lista de anexos

%\begin{listofabbrv}{UbiComp} %% Lista de abreviaturas e siglas
% \item [RJ] Rio de Janeiro
% \item [UFRuralRJ] Universidade Federal Rural do Rio de Janeiro
%\end{listofabbrv}

%\begin{listofsymbols}{teste} %% Lista de simbolos
% \item [$\varnothing$] vazio %% Simbolos devem aparecer conforme a ordem em que aparecem no texto
% \item [$\Gamma$]  Gama      %% O parâmetro deve ser o símbolo mais longo
% \item [$\forall$] Para todo
%\end{listofsymbols}

\tableofcontents %% Sumário

%%=============================================================================
%% Início da tese
%%=============================================================================

\setlength{\baselineskip}{1.5\baselineskip}

 %% Incluir capítulo 00
\artigofalse
\chapter{A FILOSOFIA DA MATEMÁTICA DE FREGE}

O objetivo deste capítulo é descrever e discutir alguns pontos da filosofia da matemática de Frege, bem como apresentar alguns resultados matemáticos desenvolvidos na terceira parte de \textit{Begriffsschrift} (1879\nocite{Frege1998}) e nas §§46-83 de \textit{Die Grundlagen der Arithmetik}.

\section{LOGICISMO}

Entre inúmeras tendências de pensamento na matemática no século XIX, uma das principais foi o movimento fundacionalista\footnote{O movimento fundacionalista, como entendemos, foi a tentativa, por parte dos matemáticos, de fornecer os fundamentos mais seguros e racionais para sua ciência.}. Muitos foram os matemáticos que exigiam um maior rigor nas definições de conceitos matemáticos e nas provas de teoremas. O movimento marcou também o rompimento entre a geometria e a aritmética. As definições de conceitos aritméticos tinham de ser explicados por meio de outros conceitos aritméticos mais básicos. Segundo alguns comentadores, por exemplo, Demopoulos (1994 \nocite{Demopoulos1995a}\nocite{Demopoulos1995}), a rigorização da matemática e o rompimento entre geometria e aritmética\footnote{Por aritmética aqui, entendemos a aritmética dos números naturais e análise real. Quando nos referirmos apenas à aritmética dos números naturais, designaremos da seguinte forma: aritmética dos números naturais.} assinalavam uma transformação nas idéias dos matemáticos, a saber, que a aritmética formava uma ciência independente. Em última análise, se a aritmética dependesse da geometria para explicar seus conceitos, então a aritmética dependeria dos conceitos de tempo e espaço\footnote{Uma outra razão para o rigor era garantir a consistência e coerência da análise.} 

\begin{quote}
Neste aspecto o combate à incursão da intuição Kantiana, as motivações intelectuais de Frege refletem as dos analistas do século XIX que buscavam livrar o cálculo e a teoria dos reais de qualquer dependência da geometria e cinemática. Assim, já em 1817 Bolzano escreveu: ‘os conceitos de tempo e movimento são tão estranhos à matemática geral quanto o conceito de espaço’.\cite[p. 76]{Demopoulos1995a}
\end{quote}
 %% Incluir capítulo 01
\artigofalse
\chapter{A FILOSOFIA NEO-FREGEANA DE CRISPIN WRIGHT}
 %% Incluir capítulo 02
\setlength{\baselineskip}{\baselineskip}

%%=============================================================================
%% Referências
%%=============================================================================

%% O arquivo 'biblio' com as referências bibliográficas deve estar no formato
%% BibTeX.

\bibliography{referencias/mybib}
\bibliographystyle{abnt}
%%=============================================================================
%% Apêndices e anexos
%% Se precisar usar alguma seção ou subseção dentro dos apêndices ou
%% anexos, utilizar o comando \tocless para não adicionar no Sumário
%% Exemplos: 
%% \tocless\section{Histórico}
%%=============================================================================

\appendix %% Apêndices
\artigofalse
\chapter{Primeiro apêndice}
\label{anex:apendiceA}

\tocless\section{Introdução}

\blindtext[2]

\tocless\section{Subseção}

\blindtext[2]
 %% Incluir apêndice A
%\include{capitulos/apendiceb} %% Incluir apêndice B

%\annex %% Anexos
%\include{capitulos/anexoa} %% Incluir anexo A
\end{document}
