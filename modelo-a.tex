%%%%%%%%%%%%%%%%%%%%%%%%%%%%%%%%%%%%%%%%%%%%%%%%%%%%%%%%%%%%%%%%%%%%%%%%%%%%%%%%
% Modelo de dissertação com texto corrido
%%%%%%%%%%%%%%%%%%%%%%%%%%%%%%%%%%%%%%%%%%%%%%%%%%%%%%%%%%%%%%%%%%%%%%%%%%%%%%%%

%% Tipo de documento e a classe a ser usada para sua formatação.
\documentclass[diss]{UFRuralRJ}

%%%%%%%%%%%%%%%%%%%%%%%%%%%%%%%%%%%%%%%%%%%%%%%%%%%%%%%%%%%%%%%%%%%%%%%%%%%%%%
%Pacote para gerar bibliografia segundo NBR6023
%%%%%%%%%%%%%%%%%%%%%%%%%%%%%%%%%%%%%%%%%%%%%%%%%%%%%%%%%%%%%%%%%%%%%%%%%%%%%%%
%\usepackage[alf]{abntex2cite}

%%==============================================================================
%% Pacotes - língua, codificação e fonte
%%==============================================================================

\usepackage[brazilian]{babel} %% use 'english' ao invés de 'brazilian' para documento escrito em inglês
\usepackage[T1]{fontenc} %% Conjunto de caracteres correto
%\usepackage{times} %% Usar fonte Adobe Times Roman, equivalente à Times New Roman
\usepackage[utf8]{inputenc} %% Para acentuação correta

%%==============================================================================
%% Pacotes - formatação de equações e números
%%==============================================================================

\usepackage{amsmath,latexsym,amssymb}
\usepackage{siunitx}                  %% Sistema Internacional de Unidades

%%==============================================================================
%% Pacotes - formatação de figuras
%%==============================================================================

%% Importar figuras corretamente
\usepackage{graphicx}

%% Diretório onde estão as figuras dos capítulos
\graphicspath{{capitulos-a/figuras/}}

\usepackage{float}
\usepackage{wrapfig}

%%==============================================================================
%% Pacotes - formatação de hyperlinks
%%==============================================================================
%% Opção 'hidelinks' disponível no pacote 'hyperref' a partir da versão 
%% 2011-02-05  6.82a. 'hidelinks' retira os retângulos do entorno das palavras
%% com links.

\usepackage[%hidelinks%, 
           bookmarksopen=true,linktoc=none,colorlinks=true,
           linkcolor=blue,citecolor=blue,filecolor=magenta,urlcolor=blue,
            pdftitle={Este título é muito legal para uma tese escrita em português},
            pdfauthor={Nome do Melhor Estudante da Rural},
            pdfsubject={Tese de Doutorado},
            pdfkeywords={LaTeX, UFRuralRJ, Documentos acadêmicos}
            ]{hyperref}

%% Definição das margens
\usepackage[inner = 30mm, outer = 20mm, top = 30mm, bottom = 20mm]{geometry}

%% Se o pacote 'hyperref' acima foi carregado, a linha abaixo corrige um bug na 
%% hora de montar o sumário da lista de figuras e tabelas. Comente a linha se o
%% pacote 'hyperref' não foi carregado.

%%=============================================================================
%% Trampa para corrigir o bug do hyperref que redefine o caption das figuras e das
%% tabelas, n�o colocando o nome ``Figura'' antes do n�mero do mesmo na lista
%%=============================================================================

\makeatletter

\long\def\@caption#1[#2]#3{%
  \expandafter\ifx\csname if@capstart\expandafter\endcsname
                  \csname iftrue\endcsname
    \global\let\@currentHref\hc@currentHref
  \else
    \hyper@makecurrent{\@captype}%
  \fi
  \@ifundefined{NR@gettitle}{%
    \def\@currentlabelname{#2}%
  }{%
    \NR@gettitle{#2}%
  }%
  \par\addcontentsline{\csname ext@#1\endcsname}{#1}{%
    \protect\numberline{\csname fnum@#1\endcsname ~-- }{\ignorespaces #2}%
  }%
  \begingroup
    \@parboxrestore
    \if@minipage
      \@setminipage
    \fi
    \normalsize
    \expandafter\ifx\csname if@capstart\expandafter\endcsname
                    \csname iftrue\endcsname
      \global\@capstartfalse
      \@makecaption{\csname fnum@#1\endcsname}{\ignorespaces#3}%
    \else
      \@makecaption{\csname fnum@#1\endcsname}{%
        \ignorespaces
        \ifHy@nesting
          \expandafter\hyper@@anchor\expandafter{\@currentHref}{#3}%
        \else
          \Hy@raisedlink{%
            \expandafter\hyper@@anchor\expandafter{%
              \@currentHref
            }{\relax}%
          }%
          #3%
        \fi
      }%
    \fi
    \par
  \endgroup
}

\makeatother

%%==============================================================================
%% Pacotes - formatação de verbatim
%%==============================================================================
%% O ambiente verbatim é o ambiente onde são inseridos exemplos de código fonte.
%% Está opção adiciona cor de fundo ao ambiente verbatim.
%% Comente para desabilitar.

\let\oldv\verbatim
\let\oldendv\endverbatim
\def\verbatim{\par\setbox0\vbox\bgroup\oldv}
\def\endverbatim{\oldendv\egroup\fboxsep0pt 
                 \noindent\colorbox[gray]{0.8}{\usebox0}\par}

%%==============================================================================
%% Pacotes - outros
%%==============================================================================

\usepackage{blindtext}                %% Amostra de texto (\blindtext[1])
%\usepackage{fix-cm} %% Tamanho das fontes da capa (já é carregado pelo pacote)

%%==============================================================================
%% Identificação do trabalho
%%==============================================================================
\titulo{Princípio de Hume: Possibilidade de uma Filosofia (Neo) Fregeana da Aritmética?} %% Título em português
\author{Duarte}{Alessandro Bandeira} %% Autor: sobrenome, nome
\instituto{Instituto de Ciências Humanas e Sociais} %% Instituto
\curso{Curso de Pós-Graduação em Filosofia} %% Curso de pós-graduação
\area{Filosofia} %% Área de concentração
\grau{Mestre em Filosofia} %% Grau obtido
\nivel{Mestrado em Filosofia} %% Nível do curso
\local{Serop\'edica}{RJ}{Brasil} %% Local da defesa

%%==============================================================================
%% Identificação dos orientadores
%%==============================================================================

\advisor[Professor]{Dr.}{Dedekind}{Richard}{UFRRJ} %% Orientador
%\orientadoratrue %% Descomentar se for uma ORIENTADORA
%\coadvisor[Pesquisador]{Dr.}{Rural}{Melhor Co-orientador da} %% Co-orientador
%\coadvisor[Professora]{Dra.}{Rural}{Melhor Co-orientadora da} %% Co-orientadora

%% COMANDOS OBSOLETOS
%% Descomentar o comando abaixo se for UMA CO-ORIENTADORA
%% No caso de DOIS OU MAIS co-orientadores, descomentar apenas quando todos forem CO-ORIENTADORAS
%\coorientadoratrue 
%\coorientadorestrue %% dois ou mais co-orientadores(as)

%%==============================================================================
%% Informações sobre a defesa
%%==============================================================================

% Ex.: \committee[Título]{Sobrenome}{Nome}{Instituição}
\committee[Dr.]{Dedekindbr}{Richard}{UFRRJ} %% Presidente
\committee[Dr.]{Frege}{Gottlob}{UFRRJ} %% Examinador
\committee[Dra.]{Maddy}{Peneloppy}{UFRGT} %% Examinador
%\committee[Dra.]{Banca}{Outra Melhor Integrante da}{MEFR} %% Examinador
%\committee[Dr.]{Banca}{Outro Melhor Integrante da}{MEFR} %% Examinador
\date{30}{Fevereiro}{2016} %% Data da defesa

%%==============================================================================
%% Início do documento
%%==============================================================================
\begin{document}
%%==============================================================================
%% Capa e folha de rosto
%%==============================================================================
\maketitle

%%==============================================================================
%% Ficha catalográfica
%%==============================================================================
%% Como a CIP vai ser impressa atrás da página de rosto, as margens 'inner' e 
%% 'outer' devem ser invertidas.
%\newgeometry{inner=20mm,outer=30mm,top=30mm,bottom=20mm}
%\makeCIP{alessandrosamuel@yahoo.com.br} %% email do autor		
%\restoregeometry
%Se for usar a catalogação gerada pelo gerador do site da biblioteca comentar as
%linhas acima e utilizar o comando abaixo
%\includeCIP{CIP.pdf}

%%=============================================================================
% Folha de aprovação
%%=============================================================================
\makeapprove

%%=============================================================================
%% Dedicatória (opcional)
%%=============================================================================

%\clearpage\mbox{}\vfill\hspace{80mm}
%\begin{minipage}{76mm}
%  \begin{flushright}
%    {\em
%    Àqueles que financiaram meus estudos...
%    \par
%    ...DEDICO!
%    }
%  \end{flushright}
%\end{minipage}

%%=============================================================================
%% Agradecimentos (opcional)
%%=============================================================================

\chapter*{Agradecimentos}

\noindent Agradeço à minha família que me apoiou em todos os momentos difíceis;\\ 
à Eleonora, pela paciência e carinho; \\
a Stefano Stival, Michael Pontes e Luciano da Silva pelas nossas conversas 
informais que edificaram muitas idéias; \\
a Flávio Esteves, por sua grande amizade; \\
ao meu grande amigo José Rubens que mesmo distante sempre teve uma palavra amiga; \\
aos Profs. Drs. Luiz Carlos P. D. Pereira e Danilo Marcondes de Souza Filho que
se dispuseram a participar da Banca Examinadora; \\
ao Prof. Dr. Marco Ruffino, pela sua amizade; \\
aos Profs. Drs. Richard Heck, Christian Thiel e Gottfried Gabriel pela 
colaboração e atenção sobre a tradução da carta de Frege a Russell (28/7/1902); \\
finalmente, ao meu orientador Prof. Dr. Oswaldo Chateaubriand pela sua pela 
orientação e paciência.

%%=============================================================================
%% Biografia (opcional)
%%=============================================================================

%\chapter*{Biografia}
%O autor nasceu, cresceu e escreveu uma tese.

%%=============================================================================
%% Epígrafe (opcional)
%%=============================================================================

\clearpage\mbox{}\vfill\hspace{80mm}\begin{minipage}{76mm}\begin{flushright}{\em
``Fazer é a melhor forma de dizer.''
\par
Autor desconhecido
}\end{flushright}\end{minipage}

%%==============================================================================
%% Resumo geral (português)
%%==============================================================================
\def\tituloportugues{Princípio de Hume: Possibilidade de uma Filosofia (Neo) Fregeana da Aritmética?}
\def\chavesportugues{Frege, Princípios de Abstração, Princípio de Hume} %% Palavras-chave em português

%\generalabstracttrue
\begin{generalabstract}{brazilian}{\tituloportugues}{\chavesportugues} %% Resumo geral em português
  A dissertação apresenta e discute as idéias desenvolvidas por Crispin Wright 
  no livro \textit{Frege’s Conception of Numbers as Objects} (1983), em 
  particular, a tese segundo a qual a aritmética é analítica. Wright deposita 
  toda sua força argumentativa (em relação à analiticidade da aritmética) na 
  derivação dos axiomas da aritmética de segunda ordem de Dedekind-Peano a 
  partir do Princípio de Hume. Assim, é nosso principal objetivo apresentar e 
  discutir em que medida o Princípio de Hume é capaz de fornecer, segundo 
  Wright, um relato da analiticidade da aritmética, assim como, as objeções a 
  esse relato.
\end{generalabstract}

%%==============================================================================
%% General abstract (inglês)
%%==============================================================================
\def\tituloingles{Hume's Principle: Possibility of a (Neo) Fregean Philosophy 
  of Arithmetic?} %% Título em inglês
\def\chavesingles{Frege, Abstraction Principles, Hume's Principle} %% Palavras-chave em inglês

%\generalabstracttrue
\begin{generalabstract}{english}{\tituloingles}{\chavesingles} %% Resumo geral em inglês
  The dissertation presents and discusses the ideas developed by Crispin Wright
  in his book \textit{Frege's Conception of Numbers as Objects} (1983), in 
  particular his thesis that arithmetic is analytic. Wright concentrates all 
  his argumentative efforts (in relation to the analyticity of arithmetic) on 
  the derivation of the axioms of Dedekind-Peano's second order arithmetic from
  Hume's Principle. Thus, it is our main goal to present and discuss how Hume's 
  Principle provides, according to Wright, an explanation of the analytic 
  character of arithmetic as well as some objections to this account.
\end{generalabstract}

%%=============================================================================
%% Listas (comentar se não houver) e sumário
%%=============================================================================

%\listoffigures %% Lista de figuras
%\listoftables %% Lista de tabelas
%\listofappendix %% Lista de apêndices
%\listofannex %% Lista de anexos

%\begin{listofabbrv}{UbiComp} %% Lista de abreviaturas e siglas
% \item [RJ] Rio de Janeiro
% \item [UFRuralRJ] Universidade Federal Rural do Rio de Janeiro
%\end{listofabbrv}

%\begin{listofsymbols}{teste} %% Lista de simbolos
% \item [$\varnothing$] vazio %% Simbolos devem aparecer conforme a ordem em que aparecem no texto
% \item [$\Gamma$]  Gama      %% O parâmetro deve ser o símbolo mais longo
% \item [$\forall$] Para todo
%\end{listofsymbols}

\tableofcontents %% Sumário

%%=============================================================================
%% Início da tese
%%=============================================================================

\setlength{\baselineskip}{1.5\baselineskip}
\setcounter{page}{1}
\artigofalse
\chapter{Introdução}
\label{chap:introduction}

O objeto de análise e discussão da presente dissertação é o agora
conhecido \textbf{Princípio de Hume}. Talvez, o leitor não esteja
familiarizado com tal nomenclatura filosófica, mas certamente, pelo
menos se conhece Frege e, principalmente, leu \textit{Die Grundlagen
der Arithmetik}\footnote{Doravante, \textbf{GLA}} (1988\nocite{Frege1988}),
reconhecerá tal princípio. Trata-se da segunda definição de número
cardinal que Frege apresenta e rejeita em \textbf{GLA}, §§62-7\footnote{No capítulo 2 da presente dissertação, discutiremos o motivo pelo
qual Frege é obrigado a rejeitar o \textbf{Princípio de Hume}.}. O Princípio de Hume tem a seguinte forma\footnote{Até onde sabemos, foi \citet{Boolos1998c} quem cunhou o nome \textbf{Princípio
de Hume}.}:

\begin{center}
$\forall F\forall G(Nx:Fx=Nx:Gx\equiv F1-1G)$
\par\end{center}

\noindent onde `$Nx:Fx$' significa o \textit{número que pertence
ao conceito F} ou, resumindo, \textit{o número de Fs} e `$F1-1G$'
significa que \textit{existe uma correlação 1-1 entre os Fs e os Gs}
(ou, como Frege diz: $F$ e $G$ são equinuméricos)\footnote{Veja Cap. 2, seção 2.5.3.}.
A leitura total do \textbf{Princípio de Hume} seria então:

\begin{description} 

\item[Princípio de Hume] \textit{para quaisquer conceitos F e G,
o número de Fs é igual ao número de Gs se e somente se os Fs estão
em uma correspondência 1-1 com os Gs}.

\end{description}

O \textbf{Princípio de Hume} é um princípio de abstração e os princípios
de abstração têm a seguinte forma:

\begin{center}
$\forall\alpha\forall\beta(\Sigma(\alpha)=\Sigma(\beta)\equiv\alpha\approx\beta)$
\par\end{center}

\noindent onde `$\Sigma...x...$' é um operador formador de termos
(termos singulares), `$\alpha$' e `$\beta$' são variáveis que percorrem
entidades de um determinado domínio original ou primitivo ( `$\alpha$'
e `$\beta$' podem percorrer objetos, conceitos de primeira ordem,
conceitos de segunda ordem, e assim por diante)\footnote{No caso do Princípio de Hume, `$\alpha$' e `$\beta$' percorrem conceitos
de primeira ordem.} e \foreignlanguage{english}{`$\approx$'} é uma relação de equivalência
(ou seja, uma relação transitiva, simétrica e reflexiva) sobre as
entidades do domínio original ou primitivo. Note que a relação de
equinumerosidade é uma relação de equivalência.

Frege, em \textbf{GLA}\footnote{§§63-67.}, formula várias instâncias
de princípios de abstração. Uma delas é o \textbf{Princípio de Hume}
apresentado acima, mas há também o \textbf{Princípio de Direção}

\begin{description}

\item[Princípio de Direção]Para quaisquer retas $a$ e $b$, a
direção da reta $a$ = a direção da reta $b$ se e somente se $a$
é paralela ou igual a $b$1\footnote{Aqui as entidades do domínio original são retas (no caso, objetos). }
(em símbolos: $\forall a\forall b(D(a)=D(b)\equiv a\parallel b)$\footnote{A relação $x$ \textit{é paralela ou igual a} $y$ é uma relação de
equivalência (sobre as entidades indicadas - retas).},

\end{description}

\noindent e o \textbf{Princípio da Forma}

\begin{description}

\item[Princípio da Forma]Para quaisquer figuras $a$ e $b$, a forma
da figura $a$ = a forma da figura $b$ se e somente se $a$ é congruente
ou igual a $b$1\footnote{A relação \textit{x é congruente ou igual a y} também é uma relação
de equivalência (sobre as entidades indicadas – figura).} (em símbolos, $\forall a\forall b(Form(a)=Form(b)\equiv a\cong b)$.

\end{description}

Mais tarde, em \textit{Grundgesetze der Arithmetik}\footnote{De agora em diante, \textbf{GGA}.}
(1962\nocite{Frege1962}), Frege apresenta um outro princípio de abstração
– a Lei Básica V

\begin{description}

\item[Lei Básica V]para quaisquer conceitos $F$ e $G$, a extensão
do conceito $F$ = a extensão do conceito $G$ se e somente se os
conceitos $F$ e $G$ são coextensionais (em símbolos, $\forall F\forall G[\{x:Fx\}=\{x:Gx\}\equiv\forall x(Fx\equiv Gx)]$.

\end{description}

O papel dos princípios de abstração é introduzir “novos objetos” (objetos
abstratos) no domínio dos objetos. Vale enfatizar a seguinte questão:
a relação de equivalência que ocorre no lado direito dos princípios
de abstração divide o domínio original das entidades (as entidades
relevantes à relação de equivalência) em classes de equivalências,
de maneira que se duas entidades pertencem à mesma classe de equivalência,
então será associado a estas entidades o mesmo “novo objeto” (objeto
abstrato). É também interessante mencionar que o operador formador
de termos `$\Sigma...x...$' pode ser entendido como uma função 1-1
entre as classes de equivalências e os novos objetos (abstratos).

Os princípios de abstração implicam que esteja associado a toda entidade
do domínio original (relevante à relação de equivalência) um objeto
abstrato. Tome, por exemplo, o \textbf{Princípio de Direção}. Como
toda reta (dado que tais retas existem) é paralela ou igual a si mesma,
ocorrerá então a seguinte situação. Temos: 
\begin{enumerate}
\item $D(a)=D(a)\equiv a\parallel a$ (uma instância do \textbf{Princípio
de Direção});
\item $a\parallel a$ (fato geométrico)
\end{enumerate}
Por lógica proposicional, segue-se, portanto, que $D(a)=D(a)$\footnote{Como afirmamos acima, o operador `$D...x...$' é um operador formador
de termos (singulares), portanto `$D(a)$' é um nome de um objeto,
no caso, um objeto abstrato. Na verdade, `$D(a)$' é referencial devido
ao princípio do contexto. Veja Cap. 3.}. E, por lógica de predicados, obtemos $\exists x(x=D(a))$.

Em última análise, o \textbf{Princípio de Direção} implica que toda
reta tem uma direção (um objeto abstrato intimamente relacionado à
reta). O mesmo ocorre com o \textbf{Princípio de Hume}, o \textbf{Princípio
da Forma} e a \textbf{Lei Básica V}, ou seja, estes princípios implicam
que todo conceito tem um número cardinal, toda figura tem uma forma
e todo conceito tem uma extensão, respectivamente\footnote{Uma vez que a relação de equivalência é reflexiva, teremos, em geral,
que $\alpha\approx\alpha$ e, portanto, $\Sigma(\alpha)=\Sigma(\alpha)$.
E, assim, $\exists x(x=\Sigma(\alpha))$.}.
 %% Incluir capítulo 00
\artigofalse
\chapter{A FILOSOFIA DA MATEMÁTICA DE FREGE}

O objetivo deste capítulo é descrever e discutir alguns pontos da filosofia da matemática de Frege, bem como apresentar alguns resultados matemáticos desenvolvidos na terceira parte de \textit{Begriffsschrift} (1879\nocite{Frege1998}) e nas §§46-83 de \textit{Die Grundlagen der Arithmetik}.

\section{LOGICISMO}

Entre inúmeras tendências de pensamento na matemática no século XIX, uma das principais foi o movimento fundacionalista\footnote{O movimento fundacionalista, como entendemos, foi a tentativa, por parte dos matemáticos, de fornecer os fundamentos mais seguros e racionais para sua ciência.}. Muitos foram os matemáticos que exigiam um maior rigor nas definições de conceitos matemáticos e nas provas de teoremas. O movimento marcou também o rompimento entre a geometria e a aritmética. As definições de conceitos aritméticos tinham de ser explicados por meio de outros conceitos aritméticos mais básicos. Segundo alguns comentadores, por exemplo, Demopoulos (1994 \nocite{Demopoulos1995a}\nocite{Demopoulos1995}), a rigorização da matemática e o rompimento entre geometria e aritmética\footnote{Por aritmética aqui, entendemos a aritmética dos números naturais e análise real. Quando nos referirmos apenas à aritmética dos números naturais, designaremos da seguinte forma: aritmética dos números naturais.} assinalavam uma transformação nas idéias dos matemáticos, a saber, que a aritmética formava uma ciência independente. Em última análise, se a aritmética dependesse da geometria para explicar seus conceitos, então a aritmética dependeria dos conceitos de tempo e espaço\footnote{Uma outra razão para o rigor era garantir a consistência e coerência da análise.} 

\begin{quote}
Neste aspecto o combate à incursão da intuição Kantiana, as motivações intelectuais de Frege refletem as dos analistas do século XIX que buscavam livrar o cálculo e a teoria dos reais de qualquer dependência da geometria e cinemática. Assim, já em 1817 Bolzano escreveu: ‘os conceitos de tempo e movimento são tão estranhos à matemática geral quanto o conceito de espaço’.\cite[p. 76]{Demopoulos1995a}
\end{quote}
 %% Incluir capítulo 01
\artigofalse
\chapter{A FILOSOFIA NEO-FREGEANA DE CRISPIN WRIGHT}
 %% Incluir capítulo 02
\setlength{\baselineskip}{\baselineskip}

%%=============================================================================
%% Referências
%%=============================================================================

%% O arquivo 'biblio' com as referências bibliográficas deve estar no formato
%% BibTeX.

\bibliography{referencias-a/biblio}
\bibliographystyle{abnt}
%%=============================================================================
%% Apêndices e anexos
%% Se precisar usar alguma seção ou subseção dentro dos apêndices ou
%% anexos, utilizar o comando \tocless para não adicionar no Sumário
%% Exemplos: 
%% \tocless\section{Histórico}
%%=============================================================================

\appendix %% Apêndices
\artigofalse
\chapter{Primeiro apêndice}
\label{anex:apendiceA}

\tocless\section{Introdução}

\blindtext[2]

\tocless\section{Subseção}

\blindtext[2]
 %% Incluir apêndice A
%\include{capitulos-a/apendiceb} %% Incluir apêndice B

%\annex %% Anexos
%\include{capitulos-a/anexoa} %% Incluir anexo A
\end{document}
