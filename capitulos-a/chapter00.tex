\setcounter{page}{1}
\artigofalse
\chapter{Introdução}
\label{chap:introduction}

O objeto de análise e discussão da presente dissertação é o agora
conhecido \textbf{Princípio de Hume}. Talvez, o leitor não esteja
familiarizado com tal nomenclatura filosófica, mas certamente, pelo
menos se conhece Frege e, principalmente, leu \textit{Die Grundlagen
der Arithmetik}\footnote{Doravante, \textbf{GLA}} (1988\nocite{Frege1988}),
reconhecerá tal princípio. Trata-se da segunda definição de número
cardinal que Frege apresenta e rejeita em \textbf{GLA}, §§62-7\footnote{No capítulo 2 da presente dissertação, discutiremos o motivo pelo
qual Frege é obrigado a rejeitar o \textbf{Princípio de Hume}.}. O Princípio de Hume tem a seguinte forma\footnote{Até onde sabemos, foi \citet{Boolos1998c} quem cunhou o nome \textbf{Princípio
de Hume}.}:

\begin{center}
$\forall F\forall G(Nx:Fx=Nx:Gx\equiv F1-1G)$
\par\end{center}

\noindent onde `$Nx:Fx$' significa o \textit{número que pertence
ao conceito F} ou, resumindo, \textit{o número de Fs} e `$F1-1G$'
significa que \textit{existe uma correlação 1-1 entre os Fs e os Gs}
(ou, como Frege diz: $F$ e $G$ são equinuméricos)\footnote{Veja Cap. 2, seção 2.5.3.}.
A leitura total do \textbf{Princípio de Hume} seria então:

\begin{description} 

\item[Princípio de Hume] \textit{para quaisquer conceitos F e G,
o número de Fs é igual ao número de Gs se e somente se os Fs estão
em uma correspondência 1-1 com os Gs}.

\end{description}

O \textbf{Princípio de Hume} é um princípio de abstração e os princípios
de abstração têm a seguinte forma:

\begin{center}
$\forall\alpha\forall\beta(\Sigma(\alpha)=\Sigma(\beta)\equiv\alpha\approx\beta)$
\par\end{center}

\noindent onde `$\Sigma...x...$' é um operador formador de termos
(termos singulares), `$\alpha$' e `$\beta$' são variáveis que percorrem
entidades de um determinado domínio original ou primitivo ( `$\alpha$'
e `$\beta$' podem percorrer objetos, conceitos de primeira ordem,
conceitos de segunda ordem, e assim por diante)\footnote{No caso do Princípio de Hume, `$\alpha$' e `$\beta$' percorrem conceitos
de primeira ordem.} e \foreignlanguage{english}{`$\approx$'} é uma relação de equivalência
(ou seja, uma relação transitiva, simétrica e reflexiva) sobre as
entidades do domínio original ou primitivo. Note que a relação de
equinumerosidade é uma relação de equivalência.

Frege, em \textbf{GLA}\footnote{§§63-67.}, formula várias instâncias
de princípios de abstração. Uma delas é o \textbf{Princípio de Hume}
apresentado acima, mas há também o \textbf{Princípio de Direção}

\begin{description}

\item[Princípio de Direção]Para quaisquer retas $a$ e $b$, a
direção da reta $a$ = a direção da reta $b$ se e somente se $a$
é paralela ou igual a $b$1\footnote{Aqui as entidades do domínio original são retas (no caso, objetos). }
(em símbolos: $\forall a\forall b(D(a)=D(b)\equiv a\parallel b)$\footnote{A relação $x$ \textit{é paralela ou igual a} $y$ é uma relação de
equivalência (sobre as entidades indicadas - retas).},

\end{description}

\noindent e o \textbf{Princípio da Forma}

\begin{description}

\item[Princípio da Forma]Para quaisquer figuras $a$ e $b$, a forma
da figura $a$ = a forma da figura $b$ se e somente se $a$ é congruente
ou igual a $b$1\footnote{A relação \textit{x é congruente ou igual a y} também é uma relação
de equivalência (sobre as entidades indicadas – figura).} (em símbolos, $\forall a\forall b(Form(a)=Form(b)\equiv a\cong b)$.

\end{description}

Mais tarde, em \textit{Grundgesetze der Arithmetik}\footnote{De agora em diante, \textbf{GGA}.}
(1962\nocite{Frege1962}), Frege apresenta um outro princípio de abstração
– a Lei Básica V

\begin{description}

\item[Lei Básica V]para quaisquer conceitos $F$ e $G$, a extensão
do conceito $F$ = a extensão do conceito $G$ se e somente se os
conceitos $F$ e $G$ são coextensionais (em símbolos, $\forall F\forall G[\{x:Fx\}=\{x:Gx\}\equiv\forall x(Fx\equiv Gx)]$.

\end{description}

O papel dos princípios de abstração é introduzir “novos objetos” (objetos
abstratos) no domínio dos objetos. Vale enfatizar a seguinte questão:
a relação de equivalência que ocorre no lado direito dos princípios
de abstração divide o domínio original das entidades (as entidades
relevantes à relação de equivalência) em classes de equivalências,
de maneira que se duas entidades pertencem à mesma classe de equivalência,
então será associado a estas entidades o mesmo “novo objeto” (objeto
abstrato). É também interessante mencionar que o operador formador
de termos `$\Sigma...x...$' pode ser entendido como uma função 1-1
entre as classes de equivalências e os novos objetos (abstratos).

Os princípios de abstração implicam que esteja associado a toda entidade
do domínio original (relevante à relação de equivalência) um objeto
abstrato. Tome, por exemplo, o \textbf{Princípio de Direção}. Como
toda reta (dado que tais retas existem) é paralela ou igual a si mesma,
ocorrerá então a seguinte situação. Temos: 
\begin{enumerate}
\item $D(a)=D(a)\equiv a\parallel a$ (uma instância do \textbf{Princípio
de Direção});
\item $a\parallel a$ (fato geométrico)
\end{enumerate}
Por lógica proposicional, segue-se, portanto, que $D(a)=D(a)$\footnote{Como afirmamos acima, o operador `$D...x...$' é um operador formador
de termos (singulares), portanto `$D(a)$' é um nome de um objeto,
no caso, um objeto abstrato. Na verdade, `$D(a)$' é referencial devido
ao princípio do contexto. Veja Cap. 3.}. E, por lógica de predicados, obtemos $\exists x(x=D(a))$.

Em última análise, o \textbf{Princípio de Direção} implica que toda
reta tem uma direção (um objeto abstrato intimamente relacionado à
reta). O mesmo ocorre com o \textbf{Princípio de Hume}, o \textbf{Princípio
da Forma} e a \textbf{Lei Básica V}, ou seja, estes princípios implicam
que todo conceito tem um número cardinal, toda figura tem uma forma
e todo conceito tem uma extensão, respectivamente\footnote{Uma vez que a relação de equivalência é reflexiva, teremos, em geral,
que $\alpha\approx\alpha$ e, portanto, $\Sigma(\alpha)=\Sigma(\alpha)$.
E, assim, $\exists x(x=\Sigma(\alpha))$.}.
